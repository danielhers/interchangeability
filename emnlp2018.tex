%
% File emnlp2018.tex
%
\documentclass[11pt,a4paper]{article}
\usepackage{acl-onecolumn}
\usepackage{latexsym}
\usepackage{url}

\title{A Clustering-based Representation of Unigrams and Concepts}

\date{}

\begin{document}
\maketitle


\section{Introduction}

Existing representations are heuristic and memory intensive
\cite{mikolov2013efficient,pennington2014glove}.
We use sequential information bottleneck clustering for unigrams and concepts
\cite{slonim2000document}.
Our code is available\footnote{\url{https://github.ibm.com/DANIELH/sib}}.

\section{Sequential information bottleneck concept clustering}

\paragraph{Extracting concept co-occurrences.}

The first step is to create a conditional distribution matrix,
of concept $x$ to occur (e.g. in a sentence) given that concept $y$ occurs, for each
$x\in X$ and
$y\in Y$\footnote{\url{https://github.ibm.com/DANIELH/sib/blob/master/cooccur/counts.py}}.

Given
\begin{enumerate}
\item concept counts (according to the TW on any corpus),
\item concept in-link counts (from Wikipedia), and
\item concept pair counts (according to the TW on any corpus),
\end{enumerate}
we first filter the list of concepts to include only concepts with at least 50
in-links,
giving us $N$ concepts.
We then define $Y$ to be the $N_1=2,000$ most frequent concepts
and $X$ to be all $N$ concepts.
We filter zero rows and columns from the resulting co-occurrence count matrix,
and remove the co-occurrence count of each concept with itself.
We create the conditional distribution $p(y|x)$ by dividing each row by its
sum\footnote{\url{https://github.ibm.com/DANIELH/sib/blob/master/sib/joint_distribution.py}},
giving a matrix of size $N\times N_1$.

\paragraph{Clustering.}

We take the $N_2=20,000$ most frequent concepts from the
conditional distribution matrix, giving a sub-matrix of size $N_2\times N_1$.
We use the following procedure to calculate $N_c=500$
clusters\footnote{\url{https://github.ibm.com/DANIELH/sib/blob/master/sib/sequential_information_bottleneck.py}}:
We repeat the following steps 10 times (``restarts'') and select the one
with the best score, defined by
\[
I(T;Y).
\]

We start by assigning each concept in $X$ randomly to one of the $T$ clusters.
We then repeatedly iterate through all concepts $x\in X$ to update their assigned
clusters, until less than 1\% of them change clusters.
At each iteration we reassign a concept $x$ to its closest cluster $t$,
where the distance is
\[
(p(x)+p(t))\cdot JS(p(y|x),p(y|t)).
\]
$JS$ is the Jensen-Shannon divergence.

The result of clustering is a $N_2 \times N_c$ matrix,
saying for each concept $x$ what its distance is from each cluster $t$.
We expand this to a $N\times N_c$ matrix by calculating concept-cluster
distances for all $N$ concepts, even those that were not used for the
sequential clustering.
This assigns a cluster for every concept.

\paragraph{Extracting concept-cluster co-occurrences.}

From the assignment to clusters we now calculate the conditional distribution
$p(t|x)$\footnote{\url{https://github.ibm.com/DANIELH/sib/blob/master/cooccur/cluster_counts.py}},
that is, the probability for there to be any concept $y$ belonging to cluster $t$
occurring in a sentence, given that we know concept $x$ occurs in the sentence.
This gives a $N\times N_c$ matrix.

\paragraph{Concept representation.}

We can now calculate a representation for each concept $x\in X$,
with several options:
\begin{itemize}
\item $u_1(x)$: its cluster ID, that is, the closest cluster by JSD.
\item $u_2(x)$: its cluster centroid, that is, $p(y|t)$ for the cluster
  $t$ closest to $x$---of dimension $N_1$.
\item $u_3(x)$: its original $p(y|x)$, of dimension $N_1$ (does not require clustering).
\item $u_4(x)$: defined as $JS(p(y|x),p(y|t))$ for each $t$---of dimension $N_c$.
\item $u_5(x)$: conditional distribution of $y$'s clusters given $x$: $p(t|x)$,
  of dimension $N_c$.
\end{itemize}

\paragraph{Measuring similarity between concepts.}

With these representations we can calculate several similarity functions
between two concepts $x_1,x_2$:
\begin{itemize}
\item $u_1(x_1)=u_1(x_2)$:
  indicates whether the concepts are in the same cluster.
\item $d(u_i(x_1),u_i(x_2))$ for $i=2,3,4,5$ and for $d$ being either
  cosine similarity or JS.
\end{itemize}


\paragraph{Measuring similarity between sentences.}

To measure the similarity between sentences $s_1,s_2$, we first use the TW
to extract the set of concepts in each sentence, $X_1,X_2$.
We then calculate the representation of a sentence $s_j$ to be the element-wise average
of the vectors $u_i(x)$ for $x\in X_j$.
We can define $u_i(s_j)$ this
way\footnote{\url{https://github.ibm.com/DANIELH/sib/blob/master/evaluation/evaluate_sentence_relatedness.py}}.

As the actual similarity function, we use $d(u_5(s_1),u_5(s_2))$ for $d$ being either
  cosine similarity or JS.
  
\paragraph{Error analysis.}
For error analysis, we use the sentence similarity benchmark,
after extracting concepts for each sentence using the
TW\footnote{\url{https://github.ibm.com/DANIELH/sib/blob/master/sentence_similarity/labeled_triplets_tw.csv}}.
This consists of rows $X_1,X_2,X_3$ where each is a set of concepts,
and by definition the second sentence was labeled as more similar to the first
sentence than the third was, that is,
\[
sim(s_1,s_2)>sim(s_1,s_3).
\]
For each row we calculate
\[
d(u_5(s_1),u_5(s_3)) - d(u_5(s_1),u_5(s_2))
\]
We want this difference to always be as big as possible,
that is, we want the third sentence to be considered more different than
the anchor than the second sentence is.
We therefore sort the list of sentence according to this value
in ascending order, and look at the first 100 items.

\bibliography{references}
\bibliographystyle{acl_natbib}


\end{document}
